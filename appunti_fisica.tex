\documentclass[a4paper,12pt]{report}

\usepackage[italian]{babel}
\usepackage{hyperref}
\usepackage{float}
\usepackage{xcolor}

\title{\textbf{Fisica}\\Appunti universitari}
\author{Luca Casadei}
\date{\today}

\begin{document}
	\maketitle
	\tableofcontents
	\chapter{Cinematica}
	Questo capitolo parla del moto dei corpi.\\
	\textbf{Punto}: Se consideriamo un punto, ci interessano le sue coordinate ${X,Y,Z}$ nello spazia, ciascuna coordinata è una funzione nel tempo:
	${X(t),Y(t),Z(t)}$ per ogni istante t il punto si troverà in una certa posizione. Questo è rappresentabile anche attraverso un vettore, che ha anch'esso 3 dimensioni.\\
	\textbf{Misura}: Le coordinate rappresentano una distanza da un'origine nello spazio.
	Nel sistema di riferimento viene rappresentata una curva in forma parametrica.
	\section{Moto rettilineo}
	Nel moto rettilineo ho una retta che ha un verso (orientata) e il punto si muove su questa retta, determiniamo con ${X(t)}$ la posizione del punto sulla retta, definito da una sola coordinata spaziale. Questa funzione è detta \textbf{legge oraria}.
	\subsection{Velocità}
	Se il corpo si sta spostando per come lo osservo, prendendo due istanti diversi ${t_1,t_2}$ il corpo è in posizioni diverse ${X_1,X_2}$, possiamo definire la velocità media come: ${V_m = \frac{\Delta_x}{\Delta_t} = \frac{X_2 - X_1}{t_2 - t_1}}$.\\
	Questa si basa su dei ${\Delta}$ macroscopici, se ${t_2}$ si avvicina a ${t_1}$, il ${\Delta}$ diventa sempre minore e il limite rappresenta effettivamente la derivata.
	\chapter{Dinamica}
	Perché un corpo si muove in un determinato modo?
\end{document} 